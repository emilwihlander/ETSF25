\documentclass[conference]{IEEEtran}
\usepackage[T1]{fontenc}
\usepackage{cite}
\usepackage{amsmath,amssymb,amsfonts}
\usepackage{algorithmic}
\usepackage{graphicx}
\usepackage{textcomp}
\def\BibTeX{{\rm B\kern-.05em{\sc i\kern-.025em b}\kern-.08em
    T\kern-.1667em\lower.7ex\hbox{E}\kern-.125emX}}
\begin{document}

\title{Volvo's Automatic Brake System}

\author{\IEEEauthorblockN{1\textsuperscript{st} Emil Wihlander}
\IEEEauthorblockA{\textit{Computer science} \\
\textit{Faculty of Engineering, LTH}\\
Lund, Sweden \\
dat15ewi@student.lu.se}
\and
\IEEEauthorblockN{2\textsuperscript{nd} Jakob H\"{o}k}
\IEEEauthorblockA{\textit{Computer science} \\
\textit{Faculty of Engineering, LTH}\\
Lund, Sweden \\
dat15jh1@student.lu.se}
}

\maketitle

\begin{abstract}
To be written.
\end{abstract}

\section{Introduction}
Autonomous cars (i. e. self-driving cars) is right around the corner whether we like it or not. Will it revolutionize the world or not? Who knows, what we do know is that the development had to start somewhere. One piece of the puzzle is to make the car have a automatic brake system. In reality, a car will automatically brake if there is an object (e. g. cyclist, fallen tree, pedestrian) in front of the vehicle to prevent accidents. 

Let's take an example. Imagine that you are driving in the suburbs when suddenly a child accidentally kicks a ball onto the streets. The kid make a run for the ball, not aware of it's surroundings. You hit the brakes as hard as you possibly can, but it's too late. The kid heavily injured. Now picture the same scenario with a automatic brake system. The car would this time be able to stop, by itself, in time and the kid is not harmed.

Mankind is constantly trying to invent devices or systems that are able to the remove the most hazardous factor, human error. Automatic braking is an example of that. 

\subsection{Volvo Cars}
\subsubsection{General information}
\subsubsection{Technical progress}

\subsection{Zenuity}
\subsubsection{General information}
\subsubsection{What's Zenuity's role in Volvo Cars} 
Zenuity Stuff\\\\
\\\textbf{Saker relaterade till kursen:}
\subsection{OSS}
Open source software (OSS) is a software open for anyone to read, modify and distribute. However depending on the licence of the OSS, tit might be more or less permissive. %TODO s. 17 i kompendiet

\subsection{Business Plan}
\subsubsection{Business Idea}
\begin{itemize}
	\item Problem
	\item Solution
	\item Benefit / Value
	\item Competition
\end{itemize}
\subsubsection{Customer Segment and Market}
\begin{itemize}
	\item Who is typical customer?
	\item How many are there?
	\item Where are they?
	\item  What are they prepared to pay for?
	\item Why do they buy?
	\item How to reach customers?
	\item How pay?
\end{itemize}
\subsubsection{Business Model}
\begin{itemize}
	\item How are you going to make a profit?
\end{itemize}
\subsubsection{Volvo's Business Plan}
\subsubsection{Business Idea}
\begin{itemize}
	\item Problem
	\item Solution
	\item Benefit / Value
	\item Competition
\end{itemize}
\subsubsection{Customer Segment and Market}
\begin{itemize}
	\item Who is typical customer?
	\item How many are there?
	\item Where are they?
	\item  What are they prepared to pay for?
	\item Why do they buy?
	\item How to reach customers?
	\item How pay?
\end{itemize}
\subsubsection{Business Model}
\begin{itemize}
	\item How are you going to make a profit?
\end{itemize}
\subsection{Patents}
\subsubsection{General Information}
\begin{itemize}
	\item Europe, Laws
	\item The US, Laws
	\item Rest of the world, Laws
\end{itemize}
\subsubsection{Volvo's Patents}
\subsection{Big Data}
Nowadays a company's big struggle is not to store all collected data, it is how to use it. The data is called "Big Data". 

Depending on the software, the collected data could be commute patterns, phone usage or as simple as the amount of user. With this kind of information the company can make smart decisions. The downside is, the more data one got, the harder it is to process. To take fully advantage of the stored information the processing velocity is key. Another problem is the variety of data a company got. What information is in reality useful? %TODO s. 79 i kompendiet

\section{Description of the system}
\begin{itemize}
	\item Mobileye http://www.mobileye.com/about/industry-firsts/
	\item Delphi https://www.delphi.com/media-old/pressreleases-old/2014/10/02/delphi-first-to-market-with-integrated-radar-and-camera-system-on-volvo-cars
	\item No open source components
	\item Ground truth, inhouse data, move to neural networks -> user data (big data).
	\item 2017 moved ADAS to new joint venture Zenuity w/ autoliv. https://www.media.volvocars.com/global/en-gb/media/pressreleases/202044/volvo-cars-and-autoliv-announce-the-launch-of-zenuity
\end{itemize}

\section{Business aspects}
\begin{itemize}
	\item Value proposition, as a differentiation <2014, now commodity.
	\item Lines up with business strategy to provide the most reliable and best safety features, part of intellisafe.
	\item Strengthens brand
	\item Zenuity (see article p. 47)
	\item Lines up with goal to be seen as part of the premium segment.
	\item Open Source Software
\end{itemize}
\section{Ethical aspects}
\begin{itemize}
	\item Data collection.
	\item incorrect brake.
	\item missed brake.
	\begin{quotation}
		City Safety should not be used to alter the way in which the driver operates the vehicle. The driver should never rely solely on this system to safely stop the vehicle.
	\end{quotation}
\end{itemize}

\section{Legal aspects}
\begin{itemize}
	\item Data collection.
	\item incorrect brake.
	\item missed brake.
	\item Open Source Software
\end{itemize}

\section{Summary}

\begin{thebibliography}{00}
	\bibitem{b1}  Mobileye. (2017) Industry Firsts. [Online]. Available: http://www.mobileye.com/about/industry-firsts/
	\bibitem{b2} Delphi. (2014, October) Delphi First to Market with Integrated Radar and Camera System on Volvo Cars. [Online]. Available: https://www.delphi.com/media-old/pressreleases-old/2014/10/02/delphi-first-to-market-with-integrated-radar-and-camera-system-on-volvo-cars
	\bibitem{b3} Volvo Car Corporation. (2017, January). [Online]. Available: https://www.media.volvocars.com/global/en-gb/media/pressreleases/202044/volvo-cars-and-autoliv-announce-the-launch-of-zenuity
	\bibitem{b4} H. Holmstr\"{o}m Olsson and J. Bosch, From ad hoc to strategic ecosystem managment: the "Three-Layer Ecosystem Strategy Model" (TeLESM). J of Soft Evolution and Process, 29(7), July 2017
	% Introduction
	% 1. Volvo Car Corporation. (2017, November). [Online]. Available: https://www.volvocars.com/intl/about/our-company/heritage
	% 2. Zenuity AB. (2017, November). [Online]. Available: https://www.zenuity.com/people-at-heart/
	% 3. Volvo Car Corporation. (2017, May). [Online]. Available: https://www.volvocars.com/intl/about/our-innovation-brands/intellisafe/autonomous-driving/news/2017/first-operating-days-for-zenuity
	% 4. M. Henley, R. Kemp, Open Source Software - An Introduction, Computer Law & Security Report, 24:77-85, 2008
\end{thebibliography}

\pagebreak
\appendix
\section*{Contribution statement}

\end{document}

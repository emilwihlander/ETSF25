\documentclass[conference]{IEEEtran}
\usepackage[T1]{fontenc}
\usepackage{cite}
\usepackage{amsmath,amssymb,amsfonts}
\usepackage{algorithmic}
\usepackage{graphicx}
\usepackage{textcomp}
\usepackage{seqsplit}
[]\def\BibTeX{{\rm B\kern-.05em{\sc i\kern-.025em b}\kern-.08em
    T\kern-.1667em\lower.7ex\hbox{E}\kern-.125emX}}
\begin{document}

\title{Volvo's Automatic Brake System}

\author{\IEEEauthorblockN{1\textsuperscript{st} Emil Wihlander}
\IEEEauthorblockA{\textit{Computer science} \\
\textit{Faculty of Engineering, LTH}\\
Lund, Sweden \\
dat15ewi@student.lu.se}
\and
\IEEEauthorblockN{2\textsuperscript{nd} Jakob H\"{o}k}
\IEEEauthorblockA{\textit{Computer science} \\
\textit{Faculty of Engineering, LTH}\\
Lund, Sweden \\
dat15jh1@student.lu.se}
}

\maketitle

\begin{abstract}
To be written.
\end{abstract}

% ---------------------------------------------------------
% ------------------- Introduction  -----------------------
% ---------------------------------------------------------

\section{Introduction}
Autonomous vehicles (i.e.\ self-driving cars) are right around the corner with Volvo Cars aiming for a 2021 release to market.\cite{ADToMarket} Will it revolutionize the world or not? Who knows, what we do know is that the development had to start somewhere. One piece of the puzzle is to make the car have an automatic brake system. 

For example, imagine a car driving in the suburbs when suddenly a child accidentally kicks a ball onto the streets. The child makes a run for the ball, not aware of its surroundings. The driver brakes, but it's due to reaction time of the driver the kid gets heavily injured. Now picture the same scenario with an automatic brake system. The car would be able to stop, by itself, in time and the accident would be avoided considering the car has (close to) no reaction time.

Another possible scenario is that the driver crashes into the rear-end of the car in front due to inattention from using their phone or changing car settings or due to reduced sight from direct sunlight or harsh weather conditions.

An automatic brake system can, with it being always on, having close to no reaction time and using multiple sensors, potentially avoid, or at least mitigate, the hazardous situations presented above.

With the driver being responsible for approximately 94\% of all car crashes, Volvo Cars, with it's three key values, Environment, Quality and Safety, see great value in reducing those types of accidents.\cite{CrashStats,VolvoValues}

\subsection{Volvo Cars}
Volvo Cars has had a strong history of leading the market when it comes to safety innovations with the three-point safety belt in 1959 and side impact protection, whiplash protection and roll-over protection in 1991, 1998 and 2002 respectively. 
While most of the innovation in the safety field up to the early -00 where protective features newer innovations focus on proactive safety such as the blind spot information system which was introduced in 2003.\cite{VolvoInnovation}

Volvo Cars has been offering an automatic brake system for rear-end collisions in it's cars since 2008 and added a similar system for pedestrians in 2010. \cite{VolvoInnovation}
These functions has since their introduction been standard in all models and was in 2015, soon after the release of the second generation of ``City Safety'', rebranded so that all their different versions of automatic braking where included in their trademark ``City Safety''.\cite{CitySafety}

With the announcement of the rebranding statistics proving the positive effect it has had on safety where provided and presented as a stepping stone towards autonomous vehicles.\cite{CitySafety}
\subsection{Zenuity}
In April 2017 a joint venture between Volvo Cars and Autoliv started its operations with the purpose to develop autonomous driving and advanced driver assist systems (ADAS).\cite{ZenuityLaunch} With both Autoliv and Volvo Cars licensing and transferring relevant intellectual property and moving personnel over to Zenuity the development of ADAS functions moved from in-house development to a separate unit. Since automatic brake systems are classified as ADAS these where most likely included in the transfer from Volvo Cars to Zenuity. 
\subsection{OSS}
Open source software (OSS) is a software open for anyone to read, modify and distribute. However depending on the licence of the OSS, tit might be more or less permissive. \cite{OSS}

\subsection{Software Patents}
Software patents are hard to grasp. From the beginning, patents were meant as a legal protection for inventors. Patents could be viewed as a reward and acknowledgement of a scientist's success, dedication and time spent on an invention. The patent itself gave the inventor monopoly of the invention and therefore protects from potential thieves who steal the idea and use it for their own purpose. \cite{SoftwarePatent} At that time, the kind of inventions would typically be a physical product such as post-office drawer lock.\cite{LockPatent}

A software program usually imply a computer program. The definition of a computer program is several lines of instruction given to a computer which will execute them sequentially. One may not patent the lines of instructions, however, in conjunction with an executing computer it can be patentable. The reasoning is that a software program needs to be part of a process and in this case an executing computer is considered a process. In Europe, The European Patent Convention (EPC) has taken the ``process'' definition a step further. \cite{SoftwarePatent}
\begin{quotation}
A computer program claimed by itself is not excluded from patentability if the program, when running on a computer or loaded into a computer, brings about, or is capable of bringing about, a technical effect which goes beyond the (normal' (sic) physical interactions between the program (software) and the computer (hardware) on which it is run. \cite[p. 36]{SoftwarePatent}
\end{quotation} 

To summarize, one can not patent the software program code itself, but with some kind of hardware it is possible.
\subsection{Big Data}
Nowadays a company's big struggle is not to store all collected data, it is how to use it. The data is called ``Big Data''. \cite{ExploitBigData}

Depending on the software, the collected data could be commute patterns, phone usage or as simple as the amount of user. With this kind of information the company can make smart decisions. The downside is, the more data one got, the harder it is to process. To take fully advantage of the stored information the processing velocity is key. Another problem is the variety of data a company got. What information is in reality useful? \cite{SpeedDataEco} % TODO: Utveckla mer

% ---------------------------------------------------------
% --------------Description of the system  ----------------
% ---------------------------------------------------------

\section{Description of the system}
The provided source describes the ``Collision Warning with Full Auto Brake and Pedestrian Detection'' system which is a complement to the first generation of ``City Safety''. The second generation (current) ``City Safety'' uses similar hardware, with the exception that the radar (and camera) is located at the top of the windscreen rather than in the grille, it will therefore be assumed that the systems are similar. \cite{SysDescription,RACam,DelphiVolvo}

The system consist of four main parts. The camera unit, the radar unit, the data fusion unit and a control unit for the automatic brake software. The control unit will communicate with the brake control unit and heads-up display control unit. The camera, radar and data fusion are all part of the unit, called ``RACam'', provided by Delphi. \cite{SysDescription,DelphiVolvo}

\subsection{RACam}
The data from the camera and radar are sent to the data fusion unit. The radar data is used for finding objects in front of the car and the distances to these objects. The camera image is used for classification of the found objects and therefore verifies whether the objects are vehicles, pedestrians, bycyclists or large animals, the unidentified objects are discarded\footnote{Which objects the ``RACam'' can identify depends which year model the car is.}. The data fusion helps reduce the risk of a false positive i.e. the car would brake without any real risks.\cite{SysDescription}
\begin{itemize}
	\item Mobileye http://www.mobileye.com/about/industry-firsts/
	\item Delphi https://www.delphi.com/media-old/pressreleases-old/2014/10/02/delphi-first-to-market-with-integrated-radar-and-camera-system-on-volvo-cars
	\item No open source components
	\item Ground truth, inhouse data, move to neural networks -> user data (big data).
	\item 2017 moved ADAS to new joint venture Zenuity w/ autoliv. https://www.media.volvocars.com/global/en-gb/media/pressreleases/202044/volvo-cars-and-autoliv-announce-the-launch-of-zenuity
\end{itemize}
% ---------------------------------------------------------
% ------------------- Business Aspects  -------------------
% ---------------------------------------------------------
\section{Business aspects}
Volvo Cars is a huge company and hence, there is a lot of business aspects to cover.

The automatic brake system is not OSS which mean the software is disclosed. This seems peculiar considering Volvo Cars history of sharing safety features. What this implies is: Volvo need differentiation. When Volvo back in the days shared the three-point safety belt it was easier to differentiate whereas today it is harder to separate one car brand from another, apart from design and trademark. Nowadays there are some ways to differentiate, e.g. price, reputation, specifications (top velocity, acceleration) or safety. This could be the reason why Volvo is not making the automatic brake system OSS, to have some kind of uniqueness and to strengthen the brand's reputation, i.e. safety first. \cite{VolvoVision}

Why is uniqueness of great importance? Lets take an example. If there were two jumpers in the same color, same price and same size, the only difference is the brand. Which jumper would one chose? In this case the brand's reputation would be the only thing that mattered. Imagine the same example, but, the brand is the same and the price is different. Now the answer is more obvious, one would purchase the cheaper jumper. Now consider this, combine the two examples above. Two jumpers with same color and same size but with different brand and price. Harder decision has to be made, pay a higher price for a brand or pay less for another. Depends on the reputation, if the pricier jumper's company has a reputation of having better quality, the extra charge might be worth it. If we apply these examples to cars instead. Volvo is trying to make the customer consider safety of the car rather then just performance or price. When a customer is considering the safety of the car, it is already a win but this how Volvo can make a profit.

\textbf{Har inte m\"ojlighet att anv\"anda OSS - Legal, business, software}

No patent could be found of Volvo's automatic brake system. This may imply multiple things. For starters, it might mean that it is hard to get a patent of such software system. The reason it is difficult to get the system approved is that it requires the software system to be a part of a piece of hardware. It is no easy task to fulfil this requirement. The software might be using hardware that is already patented or that there are no isolated hardware Volvo can patent with the software. 

Another speculation of why Volvo has no patent of the automatic brake system could be, once again, differentiation. One can not stress enough the importance of being able to separate from its competitors. A patent will force the inventor to reveal the ``recipe'' and everyone can attempt to create the same product only using a different method. Volvo does not want to share their ``secret sauce'' and therefore choose not to patent it. 

It feels tragic that a big company such as Volvo, would consider making more money over saving lives by sharing City Safety and/or making it OSS. More lives could be saved and they would further reinforce their safety image. On the flip side, if Volvo revealed City Safety the function would be more of a commodity and some customers may choose to go with another car brand. Volvo is a pioneer in safety and if they are gone, the safety progression may halt. Or who knows, it can perhaps progress instead due to the system being shared among all car manufacturers, which is one of the reasons why patent laws were introduced in the first place. That is, to be able to share knowledge and advance research.\cite{SoftwarePatent}
\iffalse
\subsubsection{Volvo's Business Plan}
\subsubsection{Business Idea}
\begin{itemize}
	\item Problem
	\item Solution
	\item Benefit / Value
	\item Competition
\end{itemize}
\subsubsection{Customer Segment and Market}
\begin{itemize}
	\item Who is typical customer?
	\item How many are there?
	\item Where are they?
	\item  What are they prepared to pay for?
	\item Why do they buy?
	\item How to reach customers?
	\item How pay?
\end{itemize}
\subsubsection{Business Model}
\begin{itemize}
	\item How are you going to make a profit?
\end{itemize}
\begin{itemize}
	\item Value proposition, as a differentiation <2014, now commodity.
	\item Lines up with business strategy to provide the most reliable and best safety features, part of intellisafe.
	\item Strengthens brand
	\item Zenuity (see article p. 47)
	\item Lines up with goal to be seen as part of the premium segment.
	\item Open Source Software
\end{itemize}
\fi

% ---------------------------------------------------------
% ------------------- Ethical Aspects  --------------------
% ---------------------------------------------------------

\section{Ethical aspects}
\begin{itemize}
	\item Data collection.
	\item incorrect brake.
	\item missed brake.
	\begin{quotation}
		City Safety should not be used to alter the way in which the driver operates the vehicle. The driver should never rely solely on this system to safely stop the vehicle.
	\end{quotation}
\end{itemize}

% ---------------------------------------------------------
% ------------------- Legal Aspects  ----------------------
% ---------------------------------------------------------

\section{Legal aspects}
No source could be found that can verify whether Volvo's automatic brake system is collecting data to a central server or not. Considering that there is no disclaimer, no information it is therefore quite likely that, currently, the system is not collecting data. No internet connection is built in by default which strengthens previous statement \cite{SensusConnect}. However, in the future, there is a great chance that Volvo will store and utilize user data.

Data collection links to Big Data, which might be essential for City Safety in the future. As mentioned, with Big Data smart decisions can be made, but collecting data such as the car's position and the video recording of its surroundings disrupt the privacy\footnote{As can be seen in figure 1 and 7 in \cite{SysDescription}.}.  Where ever the chauffeur drives, someone can be watching and take advantage of the geographical position. The owner should be aware of this when purchasing, so in a way, it is hir own choice of potentially being monitored. What if it is never disclaimed at purchase? The clueless person who is acquiring a car does not know that hir position is observed. \textbf{Couldn't find if it is illegal or not.}.

Other collected data may be the recording of its surroundings, which the pedestrian are totally unaware of. They did not get the opportunity of choosing whether they wanted being observed by cars or not. At least the buyer of the car chooses, hopefully aware, of having hir position tracked. Surprisingly this is not illegal unless recording military vehicle or a prohibited area. \textbf{add source}

If City Safety missed a critical brake, who will be responsible? According to the disclaimer on Volvo's website it is always the drivers responsibility and should never, deliberately, wait for City Safety to activate. ... \cite{CitySafetyLegal}

An incorrect brake, legally speaking, is Volvo's fault and will take full responsibility.

\begin{itemize}
	\item Data collection.
	\item incorrect brake.
	\item missed brake.
	\item Open Source Software
\end{itemize}

% ---------------------------------------------------------
% ------------------- Summary  --------------------------
% ---------------------------------------------------------

\section{Summary}

% ---------------------------------------------------------
% ------------------- References  -------------------------
% ---------------------------------------------------------

\begin{thebibliography}{00}
	\bibitem{ADToMarket} Volvocars.com. (2017). Autonomous Driving | Intellisafe | Volvo Cars. [online] Available at: https://www.volvocars.com/intl/about/our-innovation-brands/intellisafe/autonomous-driving [Accessed 12 Nov. 2017].
	\bibitem{CrashStats} NHTSA. (2015). Critical Reasons for Crashes Investigated in the National Motor Vehicle Crash Causation Survey. [online] Available at: \seqsplit{https://crashstats.nhtsa.dot.gov/Api/Public/ViewPublication/812115} [Accessed 12 Nov. 2017].
	\bibitem{VolvoValues} Volvo Car Group. (2017). Company | Volvo Car Group. [online] Available at: https://group.volvocars.com/company [Accessed 12 Nov. 2017].
	\bibitem{VolvoInnovation} Volvo Cars. (2017). Safety Innovation in Cars | Volvo Cars. [online] Available at: https://www.volvocars.com/intl/about/our-company/heritage/innovations [Accessed 12 Nov. 2017].
	\bibitem{CitySafety} Volvo Cars. (2017). Volvo Cars' standard safety technology cuts accident claims by 28 per cent. [online] Available at: \seqsplit{https://www.media.volvocars.com/global/en-gb/media/pressreleases/163733/volvo-cars-standard-safety-technology-cuts-accident-claims-by-28-per-cent} [Accessed 13 Nov. 2017].
	\bibitem{ZenuityLaunch} Autoliv. (2017). Autoliv and Volvo Cars autonomous driving joint venture Zenuity starts operations. [online] Available at: \seqsplit{http://news.cision.com/autoliv/r/autoliv-and-volvo-cars-autonomous-driving-joint-venture-zenuity-starts-operations,c2240525} [Accessed 14 Nov. 2017].
	\bibitem{OSS} M. Henley, R. Kemp, Open Source Software - An Introduction, Computer Law \& Security Report, 24:77-85, 2008
	\bibitem{ExploitBigData} J Heidrich, A Trendowicz, C Ebert. Exploiting Big Data's Benefit. IEEE Softw 33.4, pp 111-116. 2016
	\bibitem{SpeedDataEco} J. Bosch. Speed, data, and ecosystems: the future of software engineering. IEEE Software 33.1 pp. 82-88. 2016
	\bibitem{TeLESM} H. Holmstr\"{o}m Olsson and J. Bosch, From ad hoc to strategic ecosystem managment: the "Three-Layer Ecosystem Strategy Model" (TeLESM). J of Soft Evolution and Process, 29(7), July 2017
	\bibitem{SysDescription} Coelingh, E., Eidehall, A. and Bengtsson, M. (2010). Collision Warning with Full Auto Brake and Pedestrian Detection - a practical example of Automatic Emergency Braking. 13th International IEEE Conference on Intelligent Transportation Systems.
	\bibitem{RACam} Delphi. (2017). Delphi Integrated Radar and Camera System. [online] Available at: \seqsplit{https://www.delphi.com/manufacturers/auto/safety/active/racam/} [Accessed 22 Nov. 2017].
	\bibitem{DelphiVolvo} Delphi. (2014, October) Delphi First to Market with Integrated Radar and Camera System on Volvo Cars. [Online]. Available: \seqsplit{https://www.delphi.com/media-old/pressreleases-old/2014/10/02/delphi-first-to-market-with-integrated-radar-and-camera-system-on-volvo-cars}
	\bibitem{SoftwarePatent} A. Wilk, "Patentability of Software", 2012 IEEE International Conference on Software Science, Technology and Engineering, 2012.
	\bibitem{LockPatent} Yale, L. (1861). Linus Yale. 31,278.
	\bibitem{VolvoVision} Volvo Car Group. (2017). Vision | Volvo Car Group. [Online]. Available: https://group.volvocars.com/company/vision. 		[Accessed: 22- Nov- 2017].
	\bibitem{SoftwarePatent} http://ieeexplore.ieee.org/document/6236634/?part=1
	\bibitem{LockPatent} Yale, L. (1861). Linus Yale. 31,278.
	\bibitem{SysDescription} Coelingh, E., Eidehall, A. and Bengtsson, M. (2010). Collision Warning with Full Auto Brake and Pedestrian Detection - a practical example of Automatic Emergency Braking. 13th International IEEE Conference on Intelligent Transportation Systems.
	\bibitem{RACam} Delphi. (2017). Delphi Integrated Radar and Camera System. [online] Available at: \seqsplit{https://www.delphi.com/manufacturers/auto/safety/active/racam/} [Accessed 22 Nov. 2017].
	\bibitem{DelphiVolvo} Delphi. (2014, October) Delphi First to Market with Integrated Radar and Camera System on Volvo Cars. [Online]. Available: \seqsplit{https://www.delphi.com/media-old/pressreleases-old/2014/10/02/delphi-first-to-market-with-integrated-radar-and-camera-system-on-volvo-cars}
	\bibitem{SensusConnect} "Sensus Connect", Support.volvocars.com, 2017. [Online]. Available: \seqsplit{https://support.volvocars.com/uk/Pages/article.aspx?article=2f87dc53ab618edec0a801517aec401a}. [Accessed: 06- Dec- 2017].
	\bibitem{CitySafetyLegal} "City Safety™", Support.volvocars.com, 2017. [Online]. Available: \seqsplit{https://support.volvocars.com/uk/cars/pages/owners-manual.aspx?mc=y555\&my=2018\&sw=17w17\&article=f11aa8c30d28b105c0a801e8005b64ab}. [Accessed: 12- Dec- 2017]. 
	\bibitem{PrivacyPolicy} "CUSTOMER PRIVACY POLICY", Support.volvocars.com, 2017. [Online]. \seqsplit{Available: https://support.volvocars.com/uk/pages/article.aspx?article=58bd8ef320c39971c0a801513f4e18ef}. [Accessed: 06- Dec- 2017].
	\bibitem{b1}  Mobileye. (2017) Industry Firsts. [Online]. Available: http://www.mobileye.com/about/industry-firsts/
	\bibitem{b3} Volvo Car Corporation. (2017, January). [Online]. Available: \seqsplit{https://www.media.volvocars.com/global/en-gb/media/pressreleases/202044/volvo-cars-and-autoliv-announce-the-launch-of-zenuity}
	
	% Introduction
	% 1. Volvo Car Corporation. (2017, November). [Online]. Available: https://www.volvocars.com/intl/about/our-company/heritage
	% 2. Zenuity AB. (2017, November). [Online]. Available: https://www.zenuity.com/people-at-heart/
	% 3. Volvo Car Corporation. (2017, May). [Online]. Available: https://www.volvocars.com/intl/about/our-innovation-brands/intellisafe/autonomous-driving/news/2017/first-operating-days-for-zenuity
	% 4. M. Henley, R. Kemp, Open Source Software - An Introduction, Computer Law & Security Report, 24:77-85, 2008
\end{thebibliography}

\pagebreak
\appendix
\section*{Contribution statement}

\end{document}

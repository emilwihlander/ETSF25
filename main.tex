\documentclass[conference]{IEEEtran}
\usepackage[T1]{fontenc}
\usepackage{cite}
\usepackage{amsmath,amssymb,amsfonts}
\usepackage{algorithmic}
\usepackage{graphicx}
\usepackage{textcomp}
\usepackage{seqsplit}
\def\BibTeX{{\rm B\kern-.05em{\sc i\kern-.025em b}\kern-.08em
    T\kern-.1667em\lower.7ex\hbox{E}\kern-.125emX}}
\begin{document}

\title{Volvo's Automatic Brake System}

\author{\IEEEauthorblockN{1\textsuperscript{st} Emil Wihlander}
\IEEEauthorblockA{\textit{Computer science} \\
\textit{Faculty of Engineering, LTH}\\
Lund, Sweden \\
dat15ewi@student.lu.se}
\and
\IEEEauthorblockN{2\textsuperscript{nd} Jakob H\"{o}k}
\IEEEauthorblockA{\textit{Computer science} \\
\textit{Faculty of Engineering, LTH}\\
Lund, Sweden \\
dat15jh1@student.lu.se}
}

\maketitle

\begin{abstract}
To be written.
\end{abstract}

\section{Introduction}
Autonomous vehicles (i.e.\ self-driving cars) are right around the corner with Volvo Cars aiming for a 2021 release to market.\cite{ADToMarket} Will it revolutionize the world or not? Who knows, what we do know is that the development had to start somewhere. One piece of the puzzle is to make the car have an automatic brake system to assist the driver to avoid/mitigate hazardous situations. 

For example, imagine a car driving in the suburbs when suddenly a child accidentally kicks a ball onto the streets. The child makes a run for the ball, not aware of it's surroundings. The driver brakes, but it's due to reaction time of the driver the kid gets heavily injured. Now picture the same scenario with a automatic brake system. The car would be able to stop, by itself, in time and the accident would be avoided considering the car has (close to) no reaction time.

Another possible scenario is that the driver crashes into the rear-end of the car in front due to inattention.

With the driver being responsible for approximately 94\% of all car crashes, Volvo Cars with it's three key values, Environment, Quality and Safety, see great value in reducing those types of accidents.\cite{CrashStats,VolvoValues}

\subsection{Volvo Cars}
Volvo Cars has had a strong history of leading the market when it comes to safety innovations with the three-point safety belt in 1959 and side impact protection, whiplash protection and roll-over protection in 1991, 1998 and 2002 respectively. 
While most of the innovation in the safety field up to the early -00 where protective features newer innovations focus on proactive safety such as the blind spot information system which was introduced in 2003.\cite{VolvoInnovation}

Volvo Cars has been offering an automatic brake system for rear-end collisions in it's cars since 2008 and added a similar system for pedestrians in 2010. \cite{VolvoInnovation}
These functions has since their introduction been standard in all models and was in 2015, soon after the release of the second generation of ``City Safety'', rebranded so that all their different versions of automatic braking where included in Volvo's trademark ``City Safety''.\cite{CitySafety}

With the announcement of the rebranding statistics proving the positive effect it has had on safety where provided and then presented as a stepping stone towards autonomous vehicles.\cite{CitySafety}
\subsection{Zenuity}
In April 2017 a joint venture between Volvo Cars and Autoliv started its operations with the purpose to develop autonomous driving and advanced driver assist systems (ADAS) for Volvo's cars and Autoliv to sell to third-party car makers.\cite{ZenuityLaunch} With both Autoliv and Volvo Cars licensing and transferring relevant intellectual property and moving personnel over to Zenuity the development of ADAS functions moved from Volvo's in-house development to a separate unit. Since automatic brake systems are classified as ADAS these where most likely included in the transfer from Volvo Cars to Zenuity. 
\\\textbf{Saker relaterade till kursen:}
\subsection{OSS}
Open source software (OSS) is a software open for anyone to read, modify and distribute. However depending on the licence of the OSS, tit might be more or less permissive. %TODO s. 17 i kompendiet

\subsection{Business Plan}
\subsubsection{Business Idea}
\begin{itemize}
	\item Problem
	\item Solution
	\item Benefit / Value
	\item Competition
\end{itemize}
\subsubsection{Customer Segment and Market}
\begin{itemize}
	\item Who is typical customer?
	\item How many are there?
	\item Where are they?
	\item  What are they prepared to pay for?
	\item Why do they buy?
	\item How to reach customers?
	\item How pay?
\end{itemize}
\subsubsection{Business Model}
\begin{itemize}
	\item How are you going to make a profit?
\end{itemize}
\subsubsection{Volvo's Business Plan}
\subsubsection{Business Idea}
\begin{itemize}
	\item Problem
	\item Solution
	\item Benefit / Value
	\item Competition
\end{itemize}
\subsubsection{Customer Segment and Market}
\begin{itemize}
	\item Who is typical customer?
	\item How many are there?
	\item Where are they?
	\item  What are they prepared to pay for?
	\item Why do they buy?
	\item How to reach customers?
	\item How pay?
\end{itemize}
\subsubsection{Business Model}
\begin{itemize}
	\item How are you going to make a profit?
\end{itemize}
\subsection{Patents}
\subsubsection{General Information}
\begin{itemize}
	\item Europe, Laws
	\item The US, Laws
	\item Rest of the world, Laws
\end{itemize}
\subsubsection{Volvo's Patents}
\subsection{Big Data}
Nowadays a company's big struggle is not to store all collected data, it is how to use it. The data is called ``Big Data''. 

Depending on the software, the collected data could be commute patterns, phone usage or as simple as the amount of user. With this kind of information the company can make smart decisions. The downside is, the more data one got, the harder it is to process. To take fully advantage of the stored information the processing velocity is key. Another problem is the variety of data a company got. What information is in reality useful? %TODO s. 79 i kompendiet

\section{Description of the system}
\begin{itemize}
	\item Mobileye http://www.mobileye.com/about/industry-firsts/
	\item Delphi https://www.delphi.com/media-old/pressreleases-old/2014/10/02/delphi-first-to-market-with-integrated-radar-and-camera-system-on-volvo-cars
	\item No open source components
	\item Ground truth, inhouse data, move to neural networks -> user data (big data).
	\item 2017 moved ADAS to new joint venture Zenuity w/ autoliv. https://www.media.volvocars.com/global/en-gb/media/pressreleases/202044/volvo-cars-and-autoliv-announce-the-launch-of-zenuity
\end{itemize}

\section{Business aspects}
\begin{itemize}
	\item Value proposition as a differentiation.
	\item Lines up with business strategy to provide the most reliable and best safety features, part of intellisafe.
	\item Strengthens brand
	\item Zenuity
	\item Lines up with goal to be seen as part of the premium segment.
	\item Open Source Software
	\item supplier relations
\end{itemize}
\section{Ethical aspects}
\begin{itemize}
	\item Data collection.
	\item incorrect brake.
	\item missed brake.
	\begin{quotation}
		City Safety should not be used to alter the way in which the driver operates the vehicle. The driver should never rely solely on this system to safely stop the vehicle.
	\end{quotation}
\end{itemize}

\section{Legal aspects}
\begin{itemize}
	\item Data collection.
	\item incorrect brake.
	\item missed brake.
	\item Open Source Software
	\item Intellectual property
\end{itemize}

\section{Summary}

\begin{thebibliography}{00}
	\bibitem{ADToMarket} Volvocars.com. (2017). Autonomous Driving | Intellisafe | Volvo Cars. [online] Available at: https://www.volvocars.com/intl/about/our-innovation-brands/intellisafe/autonomous-driving [Accessed 12 Nov. 2017].
	\bibitem{CrashStats} NHTSA. (2015). Critical Reasons for Crashes Investigated in the National Motor Vehicle Crash Causation Survey. [online] Available at: \seqsplit{https://crashstats.nhtsa.dot.gov/Api/Public/ViewPublication/812115} [Accessed 12 Nov. 2017].
	\bibitem{VolvoValues} Volvo Car Group. (2017). Company | Volvo Car Group. [online] Available at: https://group.volvocars.com/company [Accessed 12 Nov. 2017].
	\bibitem{VolvoInnovation} Volvo Cars. (2017). Safety Innovation in Cars | Volvo Cars. [online] Available at: https://www.volvocars.com/intl/about/our-company/heritage/innovations [Accessed 12 Nov. 2017].
	\bibitem{CitySafety} Volvo Cars. (2017). Volvo Cars' standard safety technology cuts accident claims by 28 per cent. [online] Available at: \seqsplit{https://www.media.volvocars.com/global/en-gb/media/pressreleases/163733/volvo-cars-standard-safety-technology-cuts-accident-claims-by-28-per-cent} [Accessed 13 Nov. 2017].
	\bibitem{ZenuityLaunch} Autoliv. (2017). Autoliv and Volvo Cars autonomous driving joint venture Zenuity starts operations. [online] Available at: \seqsplit{http://news.cision.com/autoliv/r/autoliv-and-volvo-cars-autonomous-driving-joint-venture-zenuity-starts-operations,c2240525} [Accessed 14 Nov. 2017].
	\bibitem{b1}  Mobileye. (2017) Industry Firsts. [Online]. Available: http://www.mobileye.com/about/industry-firsts/
	\bibitem{b2} Delphi. (2014, October) Delphi First to Market with Integrated Radar and Camera System on Volvo Cars. [Online]. Available: https://www.delphi.com/media-old/pressreleases-old/2014/10/02/delphi-first-to-market-with-integrated-radar-and-camera-system-on-volvo-cars
	\bibitem{b3} Volvo Car Corporation. (2017, January). [Online]. Available: https://www.media.volvocars.com/global/en-gb/media/pressreleases/202044/volvo-cars-and-autoliv-announce-the-launch-of-zenuity
	\bibitem{b4} H. Holmstr\"{o}m Olsson and J. Bosch, From ad hoc to strategic ecosystem managment: the "Three-Layer Ecosystem Strategy Model" (TeLESM). J of Soft Evolution and Process, 29(7), July 2017
	% Introduction
	% 1. Volvo Car Corporation. (2017, November). [Online]. Available: https://www.volvocars.com/intl/about/our-company/heritage
	% 2. Zenuity AB. (2017, November). [Online]. Available: https://www.zenuity.com/people-at-heart/
	% 3. Volvo Car Corporation. (2017, May). [Online]. Available: https://www.volvocars.com/intl/about/our-innovation-brands/intellisafe/autonomous-driving/news/2017/first-operating-days-for-zenuity
	% 4. M. Henley, R. Kemp, Open Source Software - An Introduction, Computer Law & Security Report, 24:77-85, 2008
\end{thebibliography}

\pagebreak
\appendix
\section*{Contribution statement}

\end{document}
